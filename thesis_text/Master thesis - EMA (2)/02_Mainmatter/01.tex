\chapter{Introduction}
Flow cytometry is a technique useful in analysing cell populations, especially in medical fields such as immunology or hephatology. Over the past half a century, it has been perfected by many scientists to a point in which is it capable of producing so much data that it can be overwhelming. To make sense of the data, scientists and other professionals use various hierarchical clustering algorithms. It can still be challenging to find useful insights in sorted data. Data visualization techniques can help find meaning and connections that would otherwise remain hidden.

The aim of this thesis is to propose a tool which would allow the user to explore data in a user-friendly manner. The goal is to make flow cytometry data analysis accessible to those who are not well acquainted with programming.

The following chapters will introduce the reader to flow cytometry and its uses, some of the clustering algorithms that can be used on flow cytometry data, and the importance of data visualization and some of the techniques data visualization offers. 

Next, the solution to the developed tool is explored and the choices made during the process are explained. 
